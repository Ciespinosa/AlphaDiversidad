\documentclass[]{book}
\usepackage{lmodern}
\usepackage{amssymb,amsmath}
\usepackage{ifxetex,ifluatex}
\usepackage{fixltx2e} % provides \textsubscript
\ifnum 0\ifxetex 1\fi\ifluatex 1\fi=0 % if pdftex
  \usepackage[T1]{fontenc}
  \usepackage[utf8]{inputenc}
\else % if luatex or xelatex
  \ifxetex
    \usepackage{mathspec}
  \else
    \usepackage{fontspec}
  \fi
  \defaultfontfeatures{Ligatures=TeX,Scale=MatchLowercase}
\fi
% use upquote if available, for straight quotes in verbatim environments
\IfFileExists{upquote.sty}{\usepackage{upquote}}{}
% use microtype if available
\IfFileExists{microtype.sty}{%
\usepackage{microtype}
\UseMicrotypeSet[protrusion]{basicmath} % disable protrusion for tt fonts
}{}
\usepackage{hyperref}
\hypersetup{unicode=true,
            pdftitle={Medidas de Alpha Diversidad},
            pdfauthor={Carlos Iván Espinosa},
            pdfborder={0 0 0},
            breaklinks=true}
\urlstyle{same}  % don't use monospace font for urls
\usepackage{natbib}
\bibliographystyle{apalike}
\usepackage{longtable,booktabs}
\usepackage{graphicx,grffile}
\makeatletter
\def\maxwidth{\ifdim\Gin@nat@width>\linewidth\linewidth\else\Gin@nat@width\fi}
\def\maxheight{\ifdim\Gin@nat@height>\textheight\textheight\else\Gin@nat@height\fi}
\makeatother
% Scale images if necessary, so that they will not overflow the page
% margins by default, and it is still possible to overwrite the defaults
% using explicit options in \includegraphics[width, height, ...]{}
\setkeys{Gin}{width=\maxwidth,height=\maxheight,keepaspectratio}
\IfFileExists{parskip.sty}{%
\usepackage{parskip}
}{% else
\setlength{\parindent}{0pt}
\setlength{\parskip}{6pt plus 2pt minus 1pt}
}
\setlength{\emergencystretch}{3em}  % prevent overfull lines
\providecommand{\tightlist}{%
  \setlength{\itemsep}{0pt}\setlength{\parskip}{0pt}}
\setcounter{secnumdepth}{5}
% Redefines (sub)paragraphs to behave more like sections
\ifx\paragraph\undefined\else
\let\oldparagraph\paragraph
\renewcommand{\paragraph}[1]{\oldparagraph{#1}\mbox{}}
\fi
\ifx\subparagraph\undefined\else
\let\oldsubparagraph\subparagraph
\renewcommand{\subparagraph}[1]{\oldsubparagraph{#1}\mbox{}}
\fi

%%% Use protect on footnotes to avoid problems with footnotes in titles
\let\rmarkdownfootnote\footnote%
\def\footnote{\protect\rmarkdownfootnote}

%%% Change title format to be more compact
\usepackage{titling}

% Create subtitle command for use in maketitle
\providecommand{\subtitle}[1]{
  \posttitle{
    \begin{center}\large#1\end{center}
    }
}

\setlength{\droptitle}{-2em}

  \title{Medidas de Alpha Diversidad}
    \pretitle{\vspace{\droptitle}\centering\huge}
  \posttitle{\par}
    \author{Carlos Iván Espinosa}
    \preauthor{\centering\large\emph}
  \postauthor{\par}
      \predate{\centering\large\emph}
  \postdate{\par}
    \date{Noviembre de 2019}

\usepackage{booktabs}

\begin{document}
\maketitle

{
\setcounter{tocdepth}{1}
\tableofcontents
}
\chapter*{Preambulo}\label{preambulo}
\addcontentsline{toc}{chapter}{Preambulo}

El describir patrones dentro de los datos biológicos es uno de los
principales intereses de los ecólogos de comunidades. Sin embargo, las
comunidades son complejas y están caracterizadas por una gran cantidad
de variables, las especies. El rescatar los patrones a partir de esa
estructura compleja tiene varias limitantes como lo vimos en el
\href{https://ciespinosa.github.io/AnalisisMultivariante/index.html}{análisis
multivariante de la comunidad}, de esta forma el poder simplificar esta
estructura compleja a través de índices que rescaten las propiedades
emergentes de la comunidad es fundamental.

Las comunidades cambian a lo largo del paisaje como una respuesta a las
variaciones del ambiente, estos cambios en el paisaje han hecho
necesario el separar los componentes de la diversidad de la comunidad.
Según Whittaker (1972) podemos separar la diversidad en los componentes
\emph{alfa, beta} y \emph{gamma}. La diversidad \emph{\textbf{alfa}} se
refiere a la riqueza de especies que detectamos en una comunidad en un
determinado sitio más o menos homogéneo. La diversidad
\emph{\textbf{beta}} se refiere al grado de cambio o reemplazo de
especies entre diferentes comunidades en un paisaje, y la diversidad
\emph{\textbf{gamma}} se refiere a la riqueza de especies del conjunto
de comunidades que integran un paisaje y es el resultado de las
diversidades alfa y beta en el territorio (Whittaker, 1972).

\begin{quote}
La diversidad puede ser separada en diversidad alfa, beta y gamma

--- (Wittaker, 1972)
\end{quote}

La diversidad alfa está constituida por la diversidad intrínseca de una
comunidad bajo condiciones similares en un paisaje. Existen tres medidas
de alfa diversidad; riqueza, equitatividad y diversidad.

La riqueza, posiblemente la medida más sencilla, se refiere al número de
especies en un determinado sitio independiente de las abundancias de
cada una. Aunque la riqueza y los índices basados en esta son
interesantes perdemos una parte de la información en estos índices, la
abundancia.

La equitatividad se refiere a la variabilidad en las abundancias
relativas de cada una de las especies de la comunidad. Es una medida que
nos permite entender el reparto de recursos entre las especies dentro de
la comunidad, y por tanto cual es el aporte de cada una de las especies
a la comunidad.

Por otro lado, los índices de heterogeneidad están basados en la
relación entre equitatividad y riqueza. Aunque hay más de 60 índices
publicados en revistas ecológicas los índices de Shannon-Weaver y de
Simpson son los más comunes para medir alfa diversidad. La meta
fundamental detrás del diseño de la mayoría de los índices de
heterogeneidad es unificar dos elementos de la diversidad, la
equitatividad, o sea la variabilidad en las abundancias relativas de las
especies de la comunidad, y la riqueza, o sea el número total de
especies que componen la comunidad.

En el presente ejercicio intentaremos profundizar sobre estos conceptos,
las limitantes en sus cálculos así como el proceso para poder realizar
los cálculos en R.

\includegraphics{Alpha-Diversidad_files/figure-latex/unnamed-chunk-1-1.pdf}

\chapter*{Objetivos}\label{objetivos}
\addcontentsline{toc}{chapter}{Objetivos}

\begin{itemize}
\tightlist
\item
  Comprender los factores que influencian la medición de la riqueza y
  diversidad de especies.
\end{itemize}

\begin{itemize}
\tightlist
\item
  Implementar los métodos para medir y describir la riqueza y diversidad
  de las comunidades y su aplicación en el contexto de campo.
\end{itemize}

\chapter{Riqueza total del muestreo}\label{riqueza-total-del-muestreo}

Placeholder

\section{Rarefacción basada en
Individuos}\label{rarefaccion-basada-en-individuos}

\chapter{Comparando muestras}\label{comparando-muestras}

Placeholder

\chapter{Medidas de Diversidad}\label{medidas-de-diversidad}

Placeholder

\subsection{Índice de Simpson}\label{indice-de-simpson}

\chapter{Ejercicio Práctico}\label{ejercicio-practico}

Placeholder

\bibliography{book.bib}


\end{document}
